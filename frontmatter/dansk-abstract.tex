\chapter{Dansk Resumé}

\begin{otherlanguage*}{danish}


    Metoder som næste-generation sekventering i genetik og brugen af elektroniske journaler i sundhedsvæsenet har i løbet af de seneste år drastisk øget mængden af data. Nye laboratorieprotokoller har inden for arkæogenetik nu muliggjort sekventering af DNA som er millioner år gammelt. Med indførslen af elektroniske patientjournaler blev den tilgængelige mængde data øget kraftigt, hvilket har muliggjort brugen af moderne maskinlæringsmodeller. Tilsammen har disse moderne metoder dette har ført til et behov for nye værktøjer til at analysere og fortolke denne enorme mængde information -- information som ser ud til at fortsætte med at vokse i størrelse i de kommende år. Denne afhandling fokuserer på anvendelsesområder og potentielle problemer med anvendelse af moderne statistiske og datavidenskabelige metoder på forskellig biologisk data.

    Indholdet af denne afhandling er delt op i fire dele med hver sin artikel. I den første artikel introducerer vi en ny statistisk metode til analyse af DNA-skade i arkæogenetik. Vi er ikke bekendt med nogen tidligere metoder der er designet til at dække dette specifikke anvendelsesområde i genetik. Resultatet af denne forskning, \metaDMG, er et computerprogram til estimering af DNA-skade for både simple og komplekse arkæogenetiske datasæt.

    I den anden artikel præsenterer vi en ny tilgang til at forudsige medicinske komplikationer efter en knæ- eller hofteoperation ved brug af moderne maskinlæringsmodeller. Brugen af maskinlæring er stadig forholdsvis ny indenfor anæstesi og dette er et første skridt i at anvende maskinlæring indenfor dette felt. Vi viser i artiklen at moderne maskinlæringsmetoder kan anvendes til at forudsige medicinske komplikationer med høj præcision -- endda bedre end de klassisk statistiske metoder der ofte er benyttet inden for feltet. Vi viser yderligere at model-forklarings metoder ikke blot kan bruges til at forstå såkaldte modellens inderste dele, og dermed selve risikoforudsigelserne, men også kan hjælpe lægerne i deres beslutningsproces.

    Vi beskriver i den tredje artikel hvordan rumlige uensartetheder påvirker de teoretiske forudsigelser af en epidemikurve, hvis man baserer sine forudsigelser på data fra den tidlige fase af en epidemi. Vi udviklede i samarbejde med Statens Serum Institut en agent-baseret model. Denne model var bygget på de klassiske SIR-modeller som ofte er anvendt i epidemiologien. Brugen af agent-baserede modeller tillod os at modellere spredningen af sygdom i den danske befolkning og introducere komplekse interaktionsmønstre mellem agenterne i form af uensartetheder baseret på geografisk tæthed. Vi fandt at forudsigelser baseret på SIR-lignende modeller overestimerer det maksimale antal samtidig smittede, og det samlede antal smittede, med en faktor to, hvis man kun kigger på data fra den tidlige fase af en epidemi.

    Alle levende celler deler det samme DNA, dog er der stor forskel på hvilke gener som hver enkel celle rent faktisk udtrykker. Mekanismerne bag denne genregulering og dæmpningen af specifikke gener er stadig ikke forklaret fuldstændig, men man ved at den fysiske struktur af cellekernen spiller en stor rolle. Især dæmpnings- og reperationsfokusserne i cellekernen er særdeles vigtige i denne sammenhæng.  I den fjerde artikel analyserer vi disse fokusser ved hjælp af Bayesiansk inferens baseret på diffusionsmodeller. Ud fra dette måler vi diffusionskoefficienterne af fokusserne, hvilket kan bruges til at beskrive de fysiske processer som ligger til grund for skabelsen af fokusserne.


\end{otherlanguage*}