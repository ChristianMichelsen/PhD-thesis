\chapter{Dansk Resumé}

\begin{otherlanguage*}{danish}

    I løbet af de seneste år har metoder som næste-generation sekventering i genetik og brugen af elektroniske journaler i sundhedsvæsenet øget mængden af data indenfor biologien og i lægevidenskaben drastisk. Indenfor arkæogenetik har nyere laboratorieprotokoller nu muliggjort sekventering af miljø-DNA som er millioner af år gammelt. Den øgede mængde af indsamlede data i sundhedsvæsenet grundet elektroniske journaler har muliggjort brugen af moderne maskinlæringsmodeller. Tilsammen har dette har ført til et behov for nye metoder og værktøjer til at analysere og fortolke denne enorme mængde information, der synes at fortsætte med at vokse i størrelse i de kommende år. Denne afhandling fokuserer på anvendelsesområder og potentielle problemer med anvendelse af moderne statistiske og datavidenskabelige metoder på forskellig biologisk data.

    Indholdet af denne afhandling er delt over fire dele med hver sin artikel. Den første artikel introducerer en ny statistisk metode, som vi har udviklet til analyse af DNA-skade i arkæogenetik. Vi er ikke bekendt med nogen tidligere metoder, der er designet til at dække dette specifikke anvendelsesområde i genetik. Resultatet af denne artikel, \metaDMG, er et computerprogram til estimering af DNA-skade for både simple og komplekse arkæogenetiske datasæt.

    Den anden artikel præsenteres en maskinlæringstilgang til at forudsige medicinske komplikationer efter en knæ- eller hofteoperation. Brugen af maskinlæring er stadig forholdsvis ny indenfor anæstesi og dette er et første skridt i at anvende maskinlæring indenfor dette felt. Vi viser i artiklen at moderne maskinlæringsmetoder kan anvendes til at forudsige medicinske komplikationer med en høj grad af præcision -- bedre end klassiske statistiske metoder som ellers ofte er brugt i feltet. Vi viser yderligere at model-forklarings metoder ikke blot kan bruges til at forstå såkaldte modellens inderste dele, og dermed selve risikoforudsigelserne, men også kan hjælpe lægerne i deres beslutningsproces.

    Den tredje artikel beskriver hvordan rumlige uensartetheder påvirker de teoretiske forudsigelser af en epidemikurve i den tidlige fase. I samarbejde med Statens Serum Institut udviklede vi en agent-baseret model baseret på de klassiske SIR-modeller som ofte anvendes indenfor epidemiologi. Dette tillod os at modellere den danske befolkning og introducere komplekse interaktionsmønstre mellem agenterne i form af uensartetheder baseret på geografisk tæthed. Vi fandt at forudsigelser baseret på SIR-lignende modeller overestimerer det maksimuale antallet samtidig smittede og det samlede antal smittede med en faktor to, hvis man kun kigger på data fra den tidlige fase af en epidemi.

    Den fjerde artikel præsenterer en hypotese om at reperations-fokusser i cellekernen er et resultat af en polymer-bro-model. Ved at analysere enkeltmolekyledynamikker ved hjælp af Bayesiansk inferens baseret på diffusionsmodeller kan vi vise at reperations-fokusser højst sandsynligt ikke stammer fra andenordens væske-væske faseseperation, men derimod fra polymer-broer.

\end{otherlanguage*}