\chapter{Preface}
This Ph.D. thesis summarizes my scientific research in collaboration with the Niels Bohr Institute (NBI) and the Globe Institute, University of Copenhagen, and was funded by the Lundbeck Foundation. The research was supervised by Associate Professor Troels C. Petersen (NBI) and Assistant Professor Thorfinn S. Korneliussen (Globe Institute).

Being a cross-disciplinary project, the research presented in this thesis is multi-faceted and covers a wide range of topics with the main scope being the development and integration of novel statistical methods and machine learning models for the analysis of large-scale biological data. The thesis is organized as follows: First I present a brief introduction to the statistical methods and machine learning models used in the thesis and then I present the research in the form of four papers, each of which reflects a different aspect of the research. The introduction is written with my former self in mind, containing the background knowledge I would have liked to know when I started the projects. I hope that it will be useful for anyone interested in the research presented in this thesis.

The first paper presents a novel method I developed for detecting and classifying ancient DNA damage in metagenomic samples taking the full taxonomic information into account. While the first paper focuses on the development of the statistical model in the field of ancient genomics, the second paper focuses on the use of modern machine learning models in medicine and how advanced boosted decision trees can not only improve the accuracy of identifying patients at risk of being readmitted after knee or hip surgery, but doing so in a way that is interpretable as well.

In the beginning of 2020 we all experienced how COVID-19 suddenly changed our lives and impacted our societies in dramatic ways. During this time, I worked for Statens Serum Institut, the Danish Center of Disease Control, on a project to develop an agent based model capable of simulating the spread of COVID-19 in Denmark. This model is presented in the third paper and was used to inform the Danish government on how to best handle the pandemic in the early stages and the effect of contact tracing.

Lastly, in the fourth paper I show how advanced Bayesian methods can be utilized to better estimate the diffusion coefficients in silencing foci in the cell nucleus with single-particle tracking experiments.