
\chapter{Notes on the design}

This class is my personal mix of different book design influences: mainly the works of Edward R. Tufte,
% \sidecite{TufteTvdq, TufteBE, TufteEI, TufteVE} known for the big margin and the plentiness of
\parencite{heltbergSpatialHeterogeneityAffects2022a, korneliussenANGSDAnalysisNext2014} known for the big margin and the plentiness of
sidenotes and sidecaptions. The margins are however not as prominent as in Tufte's works, the main text takes a bit more space,
more like in Robert Bringhurt's typographer's bible \autocite{heltbergSpatialHeterogeneityAffects2022a}.
% \sidecite{BringhurstEoTS}%\index{Bringhurst, Robert}.

So it is a bit of a mix of Tufte and Bringhurst,
with some of my own choices for other design features, as we will see through this chapter.

\section{Document layout}

While \texttt{tufte-style-thesis} is a class for typesetting theses,
the general layout is pretty much the same as in a regular book.
A book is traditionally divided into three major sections:
the front matter, the main matter and the back matter.

% \part{This is a part}

% Here a part