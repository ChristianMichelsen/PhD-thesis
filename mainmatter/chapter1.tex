
\chapter{Notes on the design}

This class is my personal mix of different book design influences: mainly the works of Edward R. Tufte,\sidecite{TufteTvdq, TufteBE, TufteEI, TufteVE} known for the big margin and the plentiness of sidenotes and sidecaptions. The margins are however not as prominent as in Tufte's works, the main text takes a bit more space, more like in Robert Bringhurt's typographer's bible.\sidecite{BringhurstEoTS}\index{Bringhurst, Robert} So it is a bit of a mix of Tufte and Bringhurst, with some of my own choices for other design features, as we will see through this chapter.

\section{Document layout}

While \texttt{tufte-style-thesis} is a class for typesetting theses, the general layout is pretty much the same as in a regular book. A book is traditionally divided into three major sections: the front matter, the main matter and the back matter.

The \textit{front matter}\index{front matter} is for all the stuff that comes before the main content: the preface, the acknowledgements, table of contents (\textsc{toc}) and list of various types. The pages are most widely printed in roman numerals for this part. However, I personally find it confusing and a bit useless so the page numbering system is simplified for the whole book: arabic numerals starting from the very first page of the document.\footnote{Tufte and Bringhurst do full arabic in their work too, so I consider it legit.} The frontmatter still remains relevant because chapters are unnumbered, and \LaTeX{} conveniently places them in the \textsc{toc}.

The \textit{main matter}\index{main matter} is, as its name suggests, for the main content. Here everything is normal, arabic page numbering, normally numbered chapters. At the end of the mainmatter there are usually appendices, especially for scientific and technical textbooks ; for me, an appendix seems necessary in a thesis to put big figures, tables, and content that can be referred to in the main text but are too intrusive to put in the heart of the document. Chapters in the appendix are numbered with a letter to distinguish them from the main content. The main matter can also be cut in a couple of parts.\footnote{I like, for instance, to put the appendix in a dedicated part.}

The \textit{back matter}\index{back matter} is the end of the book, usually for the references part and index or glossary. Chapter numbering is turned off. At the very end, a colophon\sidecite{CarterABC} can be put to state information about the printer, publisher and stuff like that.

To sum this up, the structure of a document typeset with \texttt{tufte-style-thesis} is something like this:
\begin{itemize}
    \item title page ;
    \item front matter (dedication, abstract, acknowledgements, table of contents, list of figures, \textit{etc}) ;
    \item main matter (main content organized in numbered chapters, an appendix with letter chapters) ;
    \item back matter (references, index, glossary, colophon).
\end{itemize}
This is how \LaTeX{} books work, and how I advise to structure a document using this class. All of this is under a single page numbering system, arabics starting right from the titlepage. Eventually, this is a really heavy layout, see how the first chapter of content starts at page 18. So do not use this class unless you have a hefty content to fit all these organizing features.


\section{Page layout}

Maybe the most distinctive aspect of this class is its page layout with its big margin to put sidenotes and captions. However this is not original at all: plenty\footnote{\texttt{tufte-latex}:\\\noindent\url{https://www.ctan.org/pkg/tufte-latex},\\\texttt{classicthesis}:\\\noindent\url{https://www.ctan.org/pkg/classicthesis}, \dots} of other \LaTeX{} classes for books and theses do it just like me, and almost always better\footnote{Or in a way cleaner \LaTeX{}.} I just wanted to do my own thing here, mixing what I personally like the most in these layout types, to better learn \LaTeX{} and to really internalize this kind of design. At the end of the day it may be a more Bringhursty than Tuftey kind of look, but hey, I won't change the name of this whole thing now.

So, as you might have started to notice, the main feature of this thing is the margin, with the sidenotes\index{sidenotes}, side references, and as you will discover, side captions and everything. It has three main advantages for me:\footnote{These advantages can be seen as drawbacks for others: less space for the actually important content, irregular and somewhat unconventional design which can be harder to handle.}
\begin{itemize}
    \item it makes the main text area narrower, therefore easier to read as the line changes become smoother, sidenotes are also friendlier than footnotes ;
    \item it makes the design breathe with plenty of potential white space (when the margins are not too crowded) ;
    \item it organizes the content: non-prosaic elements are on the side, separated from the main text area which becomes less cluttered.
\end{itemize}
So this is more intended for people who like "flavoured" text: people who likes notes, parentheses, asides, \textit{etc}. It is also more suited for topics needing lots of pictures, tables, and diagrams: a novel would look terrible with this kind of layout.

Another small detail on the sidenotes, the flag of a note is in superscript in the text, but the note itself is introduced by a number in full size: this is in superscript \dots\footnote{\dots whereas this is in normal size.} This is again one of Bringhurst's advices.


\section{Headers, lists, and other content-organizing features}

The principle here is to give structuring elements which are as unobtrusive as possible, while remaining clear and easy to follow. For example, the bold headers of vanilla \LaTeX{} have been changed for more subtle italic ones. Chapters titles have been simplified to their essential parts --a number and a title-- and put as high as possible: it is completely useless to me to start a new chapter at the middle of a page.\footnote{Bringhurst roasts this kind of chapters in his \textit{Elements}: \textit{“In modern books, where the titles are shorter and the margins have been eaten by inflationary pressure, a third of the page somewhat lies vacant just to celebrate the fact that the chapter begins”}.} Though, some of space is left after the title to let it breathe a little bit ; this is a feature of Tufte's books.

The \textsc{toc}, and the other lists as well as the index and references section are thought to be that way: friendly and unobtrusive. For example, in the \textsc{toc}, the traditional dotted lines between a heading and its corresponding folio\footnote{Just flexin, folio is a fancy term for saying "page number".} is useless and unfriendly: why have the reader to follow a line with their eyes instead of just placing the page number next to the heading? So I adapted the \textsc{toc} to make it both expressive and light/minimalistic.\footnote{I find Tufte- and Bringhurst-style \textsc{toc}s too empty, at least for a thesis.} It does not support deeper headings than the section, because I think nobody looks for such detail in the table of contents.


\section{Fonts and paragraph typography}

This class has three fonts.\index{fonts}

The main text is typeset with a version of Linux Libertine,\sidecite{Libertine}\index{Linux Libertine} with enhanced math support. Here it is in \textbf{bold} and \textit{italic}.

Sans serif text, like in the titlepage, part titles and page headers (not chapter/section titles, but small reminders at the very top of the pages) are in sans serif {\sffamily Gill Sans}\index{Gill Sans}, actually Gillius, a version of Gill Sans for \LaTeX{}. Here it is in {\sffamily\bfseries bold} and {\sffamily\itshape italics}. Gill Sans is a humanist sans-serif typeface, which I find both elegant and minimalistic. It is less harsh than grotesk fonts like Helvetica or Arial.

Mono text, for code listings, is \texttt{Droid Sans Mono}\index{Droid Sans Mono}. It is smoother to my taste than the default courier-like font. Here it is in {\ttfamily\itshape italics} (unfortunately it does not support bold --yet).

The prose is organized in paragraphs indented at the first line, as it is classically seen. The first paragraph that comes after a heading, however, is not indented.\footnote{It is again an advice from Bringhurst: \textit{"The simplest way to start any block of prose is to start from the margin, flush left [\dots]."}} The text is by default not justified on the right like in Tufte's books. Apparently it makes the lines easier to recognize and follow with the eyes ; I do not find this irregularity unpleasing. But \textit{do not worry}, it can be fully justfied really easily.\footnote{I hope people have not been bummed out at by not seeing the right-justfication.}

\bgroup\justifying
For true microtypography\index{microtypography}, when the text is fully justified (like this one), the dashes, commas, points and other stuff slightly protrude in the margin to make it seem more justified than it really is.\footnote{Paradoxically, it seems more justified than when it is truly justified. See by yourself: put a ruler (or the side of the window on the right side of the text and see how the comma slightly protrudes).} For flush left text, the typesetting algorithm has also been upgraded from standard \LaTeX, reducing the line length and space width variance, and hyphenating as less as possible. Also, the spaces between small caps increase a little bit, as well as they can be increased for full caps text.
\egroup


\section{Ideas behind the design}

\index{design}These are just some thoughts I gathered that I find interesting to consider when making designs, closely or remotely.

As Antoine de Saint-Exupéry\index{de Saint-Exupéry, Antoine} once wrote:\sidecite{StExupery} \textit{"Perfection has been reached not when there is nothing left to add, but when there is nothing left to take away"}. To me, this means that minimalism is a key aspect of document design. The features and the layout must let the true content express itself: a good typography is completely transparent. That is why the design is dependent of the content: a novel and a math textbook will have completely different designs.

However, this whole Tufte-style design is far from transparent. It is easily recognizable, and people will notice the somewhat unusual design statements. Paul Rand\index{Rand, Paul} said,\footnote{Yeah, I lazily picked the two citations on the first page of the tufte-style book class showcase. Though, I find Paul Rand's a bit condescending, like, \textit{"people know nothing about good design"}.} \textit{"The public is more familiar with bad design than good design. It is in effect, conditioned to prefer bad design, because that is what it lives with. The new becomes threatening, the old reassuring"}. Edgar Tufte completely re-thought the way to display scatterplots, curves and axes, boxplots and histograms, but most people are not used to see this optimized representation, so is it a better design if most people have to give some extra effort to adapt to it ?

Then, good design must be a cultural thing. To aim perfection, one must make a blend between innovation and tradition, to be percieved as smooth as possible for the majority of people.

So, yeah, I really don't know what to think. I find --actually I hope that sidenotes and margins benefit to the reading comfort instead of ruinig it. It makes more sense when there are figures, tables and heavier stuff, but hopefully it remains relevant for prose with notes.


\part{This is a part}