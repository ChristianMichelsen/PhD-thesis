The primary content of my thesis is the four papers included in the thesis. This chapter is meant as a brief introduction to the background needed to understand the basics of the methods used throughout the papers. As such, this chapter is not meant to be a comprehensive guide to all the statistical methods and bioinformatic tools used in the papers. The original research motivation supporting the funding of this Ph.D. was multi-disciplinary and the papers included in my thesis are also highly influenced by this.

In \autoref{section:ancientDNA}, I will shortly introduce the field of ancient genomics and the statistical methods used to identify ancient DNA will be explained. Paper I, see \autoref{chapter:metadmg}, utilize modern Bayesian methods to classify which species are ancient, and which ones are not. Bayesian methods are great when possible, however, they also rely on some statistical model being defined. In the case of Paper I, the model is a beta-binomial distribution combined with a modified geometric damage profile (exponential decay).

Sometimes the model is not known and the data generation process has to be inferred by other means. This is the case in Paper II, see \autoref{chapter:hospital}, where we utilize machine learning methods to extract this information. This paper deals with estimating the individual risk scores for each patient being re-hospitalized after a knee or hip operation. \autoref{section:machine-learning} introduces the reader to basic classification with machine learning models.

While the former two papers are based on real life data, Paper III, see \autoref{chapter:covid19-agent-based-model}, concerns the development of a new agent based model for COVID-19. The model is based on the SIR model but by using an agent-based model it allows for more complex and realistic behaviour of the disease and the transmission process. The model is used to simulate the spread of virus in Denmark and to estimate the effect of contact tracing. The model is also used to simulate and predict the spread of the ``alpha'' variant of COVID-19 in Denmark. \autoref{section:agent-based-models} introduces the reader to the basics of agent based models.

Finally, the method of Bayesian model comparison of different diffusion models is introduced in Paper IV, see \autoref{chapter:diffusion}. In particular, this paper deals with different mixture-models of independent Rayleigh-distributions, and how they can be used to extract important information about the underlying diffusion processes of a polymer bridging model in cell nuclei, see \autoref{section:diffusion}.

